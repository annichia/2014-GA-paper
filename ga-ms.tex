% Use only LaTeX2e, calling the article.cls class and 12-point type.

\documentclass[12pt]{article}

% Users of the {thebibliography} environment or BibTeX should use the
% scicite.sty package, downloadable from *Science* at
% www.sciencemag.org/about/authors/prep/TeX_help/ .
% This package should properly format in-text
% reference calls and reference-list numbers.

\usepackage{scicite}

% Use times if you have the font installed; otherwise, comment out the
% following line.

\usepackage{times}
\usepackage[hyperindex, breaklinks]{hyperref} 

% The preamble here sets up a lot of new/revised commands and
% environments.  It's annoying, but please do *not* try to strip these
% out into a separate .sty file (which could lead to the loss of some
% information when we convert the file to other formats).  Instead, keep
% them in the preamble of your main LaTeX source file.
\usepackage{graphicx}

% The following parameters seem to provide a reasonable page setup.

\topmargin 0.0cm
\oddsidemargin 0.2cm
\textwidth 16cm 
\textheight 21cm
\footskip 1.0cm


%The next command sets up an environment for the abstract to your paper.

\newenvironment{sciabstract}{%
\begin{quote} \bf}
{\end{quote}}


% If your reference list includes text notes as well as references,
% include the following line; otherwise, comment it out.

\renewcommand\refname{References and Notes}

% The following lines set up an environment for the last note in the
% reference list, which commonly includes acknowledgments of funding,
% help, etc.  It's intended for users of BibTeX or the {thebibliography}
% environment.  Users who are hand-coding their references at the end
% using a list environment such as {enumerate} can simply add another
% item at the end, and it will be numbered automatically.

\newcounter{lastnote}
\newenvironment{scilastnote}{%
\setcounter{lastnote}{\value{enumiv}}%
\addtocounter{lastnote}{+1}%
\begin{list}%
{\arabic{lastnote}.}
{\setlength{\leftmargin}{.22in}}
{\setlength{\labelsep}{.5em}}}
{\end{list}}


% Include your paper's title here

\title{Troubleshooting Pertussis Vaccination: One Booster Schedule Doesn't Fit All} 


% Place the author information here.  Please hand-code the contact
% information and notecalls; do *not* use \footnote commands.  Let the
% author contact information appear immediately below the author names
% as shown.  We would also prefer that you don't change the type-size
% settings shown here.

\author
{Maria A Riolo,$^{1,2\ast}$ Pejman Rohani$^{1,3,4}$\\
\\
\normalsize{$^{1}$Center for the Study of Complex Systems,}\\
\normalsize{$^{2}$Department of Mathematics,}\\
\normalsize{$^{3}$Department of Ecology \& Evolutionary Biology,}\\
\normalsize{University of Michigan, Ann Arbor, MI 48109, USA}\\
\normalsize{$^{4}$Fogarty International Center, National Institutes of Health,}\\
\normalsize{Bethesda, MD 20892, USA}\\
\\
\normalsize{$^\ast$To whom correspondence should be addressed; E-mail:  mariolo@umich.edu.}
}

% Include the date command, but leave its argument blank.

\date{}



%%%%%%%%%%%%%%%%% END OF PREAMBLE %%%%%%%%%%%%%%%%



\begin{document} 

% Double-space the manuscript.

\baselineskip24pt

% Make the title.

\maketitle 



% Place your abstract within the special {sciabstract} environment.

\begin{sciabstract}
  


Pertussis has become a major public health concern in many countries
where it was once a plausible candidate for vaccine-based elimination.
Although the mechanisms behind these resurgences remain elusive,
many countries have nevertheless recommended additional
booster vaccinations against pertussis, the timing and number of which vary widely.
We searched for cost-effective booster vaccination strategies
using a genetic algorithm
and found that which booster schedules most successfully controlled pertussis transmission
strongly depended on why pertussis transmission had persisted even with routine infant vaccination.
Our results suggest that booster vaccination schedules are most effective when tailored to the problem at hand
and that that misidentifying the causes of pertussis resurgences may be costly.
\end{sciabstract}



% In setting up this template for *Science* papers, we've used both
% the \section* command and the \paragraph* command for topical
% divisions.  Which you use will of course depend on the type of paper
% you're writing.  Review Articles tend to have displayed headings, for
% which \section* is more appropriate; Research Articles, when they have
% formal topical divisions at all, tend to signal them with bold text
% that runs into the paragraph, for which \paragraph* is the right
% choice.  Either way, use the asterisk (*) modifier, as shown, to
% suppress numbering.


Reconciling the recent resurgences of pertussis in many highly-vaccinated countries
with
the initial successes of routine pertussis vaccination programs
and the global heterogeneity in recent trends of pertussis incidence
has proved difficult~\cite{Jackson&Rohani:2013}, 
and a wide variety of competing explanations have been proposed.
There are many mechanisms which could contribute to increases in pertussis incidence,
including improvement in the surveillance and diagnosis of pertussis~\cite{Cherry:2003, Cherry:2012}
or reduction of the protection afforded by vaccination, 
whether due to the evolution of \textit{Bordetella pertussis}~\cite{Mooi_et:2001} or the
the switch from whole cell to acellular vaccines~\cite{Shapiro:2012}.
Even without any changes in the nature of pertussis transmission, control, or reporting,
a resurgence of pertussis might be expected in some countries,
as a simple mathematical consequence of a history of insufficient vaccination~\cite{Riolo_et:2013}.

Disentangling the many pathways to pertussis resurgence is particularly difficult
because pertussis immunity is not well-understood. 
With no known, reliable markers of pertussis immunity~\cite{Farizo_et:2014}, 
the properties of infection- and vaccine-derived protection against pertussis
must be inferred indirectly.
However, the models of pertussis immunity that best fit
individual level clinical data and population level incidence data, respectively,
are strikingly different.
While 
the data from reinfection studies~\cite{Jenkinson:1988} and animal models~\cite{Warfel_et:2014}
can be explained by
a vaccine that protects against disease for a limited duration
and may not protect against transmission at all,
large scale pertussis incidence data
are more consistent with
long-lasting vaccine-derived immunity and 
sufficiently little transmission of pertussis among
vaccinated and previously infected individuals
as to confer some protection on unvaccinated individuals 
via herd-immunity~\cite{Nielsen_et:1994,Wearing&Rohani:2009, Blackwood_et:2013}.

Despite the uncertainty surrounding the properties of pertussis vaccines, 
several countries experiencing increased pertussis incidence
have supplemented existing infant vaccinations with additional
booster vaccinations \cite{Zepp_et:2011}.
We investigated which booster schedules afforded the greatest reduction in disease burden
for the least logistical and monetary cost. 
Because the causes of the pertussis resurgence remain unclear, 
and need not be the same in every location, 
we consider four scenarios 
in which infant vaccination alone may fail to eliminate pertussis:


\paragraph{Scenario I: Insufficient vaccine coverage.}
Perhaps the simplest reason a disease may persist despite infant vaccination
is if too few infants are vaccinated to curtail transmission.
This scenario would describe, for example, 
Thailand and Italy for several decades after the introduction of pertussis vaccines,
where an increase in vaccine coverage, such as has occurred in the past three decades in both countries,
has the potential to substantially reduced pertussis incidence.
In many other regions, low vaccine coverage is likely to remain an important factor in 
the persistence of pertussis transmission.
As of October 2013, WHO estimates of vaccine coverage suggest that in \%10 of countries for which vaccination data is available
at most \%80 of infants receive the recommended three doses of pertussis vaccine~\cite{WHO:2013}.
Even in nations with high overall vaccination rates, 
communities with high concentrations of unvaccinated or undervaccinated children
may be at risk of outbreaks, as seems to be the case in some regions of the US~\cite{Omer:2009}.

\paragraph{Scenario II: Low efficacy.}
Primary vaccine failure, where a vaccine sometimes fails to provide protection when administered,
is another mechanism by which a disease might persist despite infant vaccination efforts.
For example, the outbreaks of pertussis in Canada beginning in the late 1980s
have largely been attributed to the 
poor efficacy of the Connought Laboratories adsorbed vaccine
that was used between 1985 and 1998~\cite{Halperin_et:1989, Ntezayabo_et:2003}.
More generally, the lack of serological measures of immunity to pertussis
makes it very difficult to determine the rates of primary vaccine failure,
much less distinguish such failures in ``take'' from other phenomena such as 
waning immunity.

\paragraph{Scenario III: Waning immunity.}
Even if vaccination is widespread and initially efficacious, 
infant vaccination alone may fail to achieve elimination if vaccine derived immunity is only temporary.
Some waning of vaccine derived immunity to pertussis is supported by
data at both the population and individual scales,
but while population level patterns of incidence generally suggest long-lasting immunity~\cite{Wearing&Rohani:2009},
individual level data on reinfections is better explained by a much shorter duration of protection~\cite{Jenkinson:1988}.

\paragraph{Scenario IV: leaky immunity.}
One potential mechanism for the changing epidemiology of pertussis
is the evolution of \textit{B. pertussis}~\cite{Mooi_et:2001}.
In the case of multiple  circulating strains of pertussis,
for which vaccination induces at most limited cross-immunity,
the probability of developing a transmissible, symptomatic infection
after contact with an infectious individual may be reduced (but not eliminated) by vaccination.
Such ``leakiness''~\cite{Halloran_et:1992} in vaccine derived immunity is another mechanism 
with the potential to facilitate the persistence of pertussis
despite infant vaccination.

For simplicity, our current study considers only the extreme cases 
where one of these problems alone is responsible for ongoing pertussis transmission.
However, finding successful, cost-effective booster schedules remains a non-trivial task. 
Because the space of possible schedules is 
so high-dimensional and the dynamics of infectious diseases are so nonlinear,
many traditional optimization tools are ill-suited for the problem.

Here, we use a genetic algorithm (GA) to evolve cost-effective booster schedules
by simulating the operation of natural selection on 
a ``population'' of schedules for booster vaccination.
Each strategy in the population has a genotype that encodes its
prescribed schedule of booster vaccinations.
The choice of how to encode strategies
can have important effects on the performance of a genetic algorithm, 
affecting both linkage and epistasis between genes
as well as the topographical features of the fitness space.
In the current study, we encode booster schedules on a
single chromosome containing the
probability of vaccination for each age cohort, to be
carried out at the start of the each school term.

\begin{figure}[h!]
\caption{Our Super Cool GA Figure }
\label{fig:GA}
\end{figure}


To evaluate the fitness of a strategy, we 
simulate transmission dynamics following the introduction
of the booster schedule using an age-structured SEIR model.
Individuals are born susceptible, with routine infant vaccination
occurring at five months of age.
Susceptible individuals become exposed through contact with
infectious individuals and, after a latent period, become infectious.
After recovering, individuals are immune to further infection. In the current study,
we consider infection derived immunity to be lifelong, based on
previous studies of pertussis in Sweden and Thailand, which 
found that previously infected individuals contributed very little to transmission
~\cite{Rohani_et:2010, Blackwood_et:2013}.
The number of individuals of each age in each state (susceptible, exposed, infectious,
recovered, and vaccinated) is updated stochastically with a fixed time step of one week.

Because the past history of pertussis incidence and infant vaccination are likely to have long-lasting effects on 
ongoing transmission dynamics\cite{Riolo_et:2013}, we simulated pre-booster conditions with the same history of vaccination
and chose parameters of coverage, efficacy, duration, and leakiness
to give the same overall disease burden (measured in DALYs) 
in the two decades preceding the booster campaign (Fig. \ref{fig:IC}A-D).

\begin{figure}

\begin{tabular}{ll}
A&B\\
\includegraphics[width=85mm]{"code/figs/1-ICs-coverage"}&
\includegraphics[width=85mm]{"code/figs/1-ICs-efficacy"} \\
C&D\\
\includegraphics[width=85mm]{"code/figs/1-ICs-waning"}&
\includegraphics[width=85mm]{"code/figs/1-ICs-leakiness"}\\
\end{tabular}

\caption{\footnotesize Pre-boosting incidence (solid red line) and infant vaccine uptake (dashed blue line)
 with 70\% coverage (A), 70\% efficacy (B),
a 45 year mean duration of immunity (C),  
leaky immunity preventing 85.5\% of infections (D). In each case, 
we simulated 100 years pre-vaccine (plotted starting at year 50).
In year 100, vaccination began at 60\% of its eventual coverage level 
and increased linearly over 20 years. 
}

\label{fig:IC}
\end{figure}



We can think of the simulated transmission dynamics 
under a booster schedule as the
phenotype of that strategy, with
the combined cost of infections and vaccinations
incurred during the simulation determining the strategy's fitness. 
We assessed the cost of infections based on the expected
disability-adjusted life-years (DALYs) lost, calculated as
the expected duration of illness
plus, in the case of a death,
the expected years of life lost
(average lifespan minus the age at death).
For simplicity, we assume that the costs of vaccination scale
linearly with the number of doses given, rather than
considering aspects of vaccination efdisabifort which
may be nonlinear.

To generate a new population of booster schedules, 
parent schedules are chosen using ``tournament selection''\cite{Goldberg&Kalyanmoy:1991},
in which pairs of schedules are selected uniformly
at random with replacement 
to compete for the opportunity to reproduce.
In each tournament, more fit schedule of the pair has a higher probability 
(in the current study, 90\%) of winning.
Carrying out $N$ such tournaments,
where $N$ is the population size (in the current study, 2000 individuals), yields a pool of $N$ parents.
Note that while more fit strategies are likely to appear more often
as parents than less fit strategies, a strategy's reproductive opportunities
are only determined by its rank.
This has the benefit of being less sensitive to the particular fitness function
used than a selection method relying on the actual fitness values of strategies
would be.

Once $N$ parents have been selected, these parents are divided into pairs and 
each (not necessarily unique) pair of parents 
produces a pair of offspring.
The parent chromosomes undergo 
crossover at a randomly chosen point
along the chromosome.
In each child chromosome,
point mutations occur at a fixed rate $\mu$ 
(set to 0.05 per site in our experiments,
which yields an average of 1.25 mutations per chromosome).
This process leaves us with a new generation of booster strategies, 
ready to be evaluated through simulation.

Visualizing the evolving population of booster schedules presents its own challenges.
While an average strategy can provide a rough idea of a genetic algorithm's behavior,
what we are often interested in is more akin to a ``typical good strategy,''
particularly if our algorithm may have discovered multiple, comparably fit strategies.
Continuing the analogy of booster schedules as genotypes,
we are interested in something like the consensus sequence of the most successful
species-like clusters of schedules. 
To this end, we constructed a nearest-neighbor
network of the most successful schedules found by the GA
and applied a spectral community detection algorithm 
to find clusters in the network~\cite{Riolo&Newman:2012}.
This allows us to visualize
the distribution of alleles within each cluster of schedules
in a more intuitive way, such as looking at the 
interquartile range of booster coverage in each age group,
without losing as much information about the correlation between traits
as we would by looking at the same summary statistics over the whole population.


\begin{figure}[h!]

\begin{tabular}{ll}
A&B\\
\includegraphics[height=70mm]{"code/figs/2-fitness-costs-coverage"} &
\includegraphics[height=70mm]{"code/figs/2-fitness-costs-efficacy"} \\
C&D\\
\includegraphics[height=70mm]{"code/figs/2-fitness-costs-waning"} &
\includegraphics[height=70mm]{"code/figs/2-fitness-costs-leakiness"} \\
\end{tabular}

\caption{\footnotesize Fitness costs by algorithm generation. 
Density maps of the fitness costs (combined vaccination effort and disease burden)
of the 2000 strategies in each generation of the algorithm for
(A) Scenario I: 70\% infant coverage,
(B) Scenario II: 70\% vaccine efficacy
(C) Scenario III: 45 year mean duration of vaccine derived immunity in (G,H,I),
and 
(D) Scenario IV: 14.5\% leakiness.
In each case, darker color indicates a higher density of strategies.
}
\end{figure}


In all four scenarios, the GA rapidly converges on a population of booster schedules
with lower combined vaccination and disease costs than the initial, randomly generated
population of strategies (Fig. \ref{fig:ByGeneration}).
When vaccine coverage of infants was insufficient (Scenario I),
the successful booster schedules found by our GA had, on average,
very little vaccination in older age groups, focusing most of their vaccination effort on 
young children (\ref{fig:Best}A), which
achieved more reduction of disease for less vaccination effort 
than the initial random population of strategies
 (Fig. \ref{fig:BreakdownByGeneration}A). 
 
 

\begin{figure}[h!]
\begin{tabular}{ll}
A&B\\
\includegraphics[width=85mm]{"code/figs/3-families-coverage"}
&
\includegraphics[width=85mm]{"code/figs/3-families-efficacy"}  \\
C&D\\
\includegraphics[width=85mm]{"code/figs/3-families-waning"} &
\includegraphics[width=85mm]{"code/figs/3-families-leakiness"}\\
\end{tabular}

\caption{\footnotesize Best booster schedules found for 
70\% infant coverage (A), 70\% efficacy (B),
a 45 year mean duration of immunity (C),  
leaky immunity preventing 85.5\% of infections (D).
The best 500 booster schedules found in the last algorithm generation of each scenario are arranged into a network
where each schedule is connected to the nearest 25 (under $L_0$ distance between genomes) strategies.
The strategies are the clustered using a spectral partitioning algorithm (cluster indicated in the figure by color).
The large plot shows the interquartile radius within each cluster of coverage at each age group (colors
in plot correspond to colors in network). The inset plot shows the fitness costs (in DALYs) of strategies in each family,
ordered from most fit (red, left) to least fit (purple, right).
}

\label{fig:Best}
\end{figure}



For an infant vaccination program with low coverage (Scenario II), we found five distinct clusters of strategies,
each corresponding to adding a single booster at 1, 2, 3, 4, or 5 years of age.
Our results for a low efficacy vaccine were very similar (Fig. \ref{fig:ByGeneration}B, Fig. \ref{fig:Best}B),
with the exception that a higher vaccination effort was necessary to achieve the same effective 
coverage from the booster vaccine.
In both cases, the most effective strategies came in the form of a single pre-school booster,
the timing of which made little difference to fitness as long as it came before 
children entered school, with the high contact-rates and strong age-assortative mixing
that come with it (Fig. \ref{fig:Susceptibles}A,B).

Booster schedules for a waning vaccine (Scenario III) 
differ greatly from those evolved to deal with low coverage or low efficacy (Fig. \ref{fig:Best}C).
The most successful schedules 
call for one booster in the late teens and a second booster anytime between 35 and 45 years of age
(with the exception of the strategy group shown in purple which does not include a single teenage booster
but would, on average, be expected to include one booster between the ages of 15 and 35).
These differences can be understood as a difference in the age-distribution of susceptible individuals.
When we compare booster schedules evolved for vaccines with other durations of immunity (Fig. \ref{fig:waning}),
we see that, although the number and timing of vaccinations depend on the rates of waning,
successful schedules maintain vaccine-derived protection in large fraction of the population
aged 5-45 years, particularly the school-aged children.



\begin{figure}[h!]
%\includegraphics[width=0.9\textwidth]{"code/Sfigs/S5-waning-by-time"}\\

\includegraphics[width=\textwidth]{"code/Sfigs/S5-vaccinated-mid-year"}


\caption{\footnotesize 
The fraction of individuals protected by vaccination in each age group,
plotted for varying durations of vaccine derived immunity.
In each case, the shaded region represents the interquartile range of the top performing
quartile of booster schedules in the last generation of the genetic algorithm.
}

\label{fig:waning}
\end{figure}

Leaky immunity (Scenario IV) presents yet another picture. 
The GA finds no booster schedules that effectively reduce the
disease burden (Fig. \ref{fig:BreakdownByGeneration}J).
Looking at the nearest-neighbor network of the most successful strategies, there is no discernible community structure
and the best strategies eschew booster vaccination \ref{fig:Best}D). Again, the explanation lies in the patterns of susceptibility
in the population. With infant vaccination already granting leaky protection to everyone, 
any remaining transmission already occurs among vaccinated individuals
and cannot be disrupted by additional booster vaccinations (Fig. \ref{fig:Susceptibles}D).
(Results for the scenario where booster vaccinations cause immunity to become less leaky are presented in SOM and
resemble the results for a low level of primary vaccine failure.)



Our model is relatively simple and leaves out many mechanisms
that may play important roles in pertussis transmission and control, including the
spatial or social aggregation of unvaccinated individuals~\cite{Omer:2009}, 
the occurrence of asymptomatic cases of pertussis~\cite{deGreeff_et:2010},
and household structure~\cite{deGreeff_et:2010}.
Perhaps equally importantly, our model does not attempt to mimic the real vaccination history and demographics
of any place, instead exploring the relatively simple case of a linear increase in vaccine uptake followed by sustained,
high levels of infant vaccination.


\begin{figure}[h!]
\includegraphics[width=\textwidth]{"code/figs/5-xtest"}
\caption{\footnotesize The fitness cost (combined vaccination cost and disease burden)
applying a boosting schedule evolved for each scenario 
in each other scenario.
The representative boosting schedules are constructed by taking the mean coverage at each age
in the most fit strategy families. 
These representative schedules (Strategy used) are applied to each scenario (Actual scenario) and the mean fitness cost
among 120 replicant runs is plotted.
}

\label{fig:XTest}
\end{figure}

While our results do not predict the quantitative impact of any particular vaccine schedule on any particular population,
they nevertheless have implications for vaccine policy.
We find that optimal booster schedules for controlling pertussis differ 
greatly for differing mechanisms of vaccine failure and that,
for example, a booster vaccine schedule optimized to compensate for primary vaccine failure 
may be extremely ineffective in controlling a pertussis resurgence caused by 
waning immunity and \textit{vice versa} (Fig. \ref{fig:XTest}).
Our results suggest that understanding pertussis immunity 
is critical to developing cost-effective control strategies.


\bibliography{ga}

\bibliographystyle{Science}



% Following is a new environment, {scilastnote}, that's defined in the
% preamble and that allows authors to add a reference at the end of the
% list that's not signaled in the text; such references are used in
% *Science* for acknowledgments of funding, help, etc.

%\begin{scilastnote}
%\item We've included in the template file \texttt{scifile.tex} a new
%environment, \texttt{\{scilastnote\}}, that generates a numbered final
%citation without a corresponding signal in the text.  This environment
%can be used to generate a final numbered reference containing
%acknowledgments, sources of funding, and the like, per {\it Science\/}
%style.
%\end{scilastnote}




% For your review copy (i.e., the file you initially send in for
% evaluation), you can use the {figure} environment and the
% \includegraphics command to stream your figures into the text, placing
% all figures at the end.  For the final, revised manuscript for
% acceptance and production, however, PostScript or other graphics
% should not be streamed into your compliled file.  Instead, set
% captions as simple paragraphs (with a \noindent tag), setting them
% off from the rest of the text with a \clearpage as shown  below, and
% submit figures as separate files according to the Art Department's
% instructions.


\clearpage

%\noindent {\bf Fig. 1.} Please do not use figure environments to set
%up your figures in the final (post-peer-review) draft, do not include graphics %in your
%source code, and do not cite figures in the text using \LaTeX\
%\verb+\ref+ commands.  Instead, simply refer to the figure numbers in
%the text per {\it Science\/} style, and include the list of captions at
%the end of the document, coded as ordinary paragraphs as shown in the
%\texttt{scifile.tex} template file.  Your actual figure files should
%be submitted separately.



\setcounter{figure}{0}
\renewcommand{\thefigure}{S\arabic{figure}}


\setcounter{page}{1}
\renewcommand{\thepage}{S\arabic{page}}


%\lfoot{} \cfoot{\myfont \thepage\ of \pageref{LastPage}} \rfoot{}



\section*{Supplementary Information}

\subsection*{Calculating strategy fitness}

Although our transmission model does not account for mortality, 
for the purposes of assessing costs
we assume a case fatality rate of 
0.2\% in infants one year or younger,
of 0.04\% in children 1-4 years of age\cite{},
and no mortality in older age groups. In all age groups,
we assume that both symptoms and infectiousness 
last, on average, for 15 days.
Thus, under our parameterization,
a case in a six month old costs
$\frac{15}{365} + 0.002 * 74.5 \approx 0.19$ DALYs,
while a case in an adult costs only 
$\frac{15}{365} \approx 0.041$ DALYs.

Throughout the study, we used a cost of 
\$33 per dose of vaccine and
\$50000 per DALY.
However, it is worth noting that in our current model
changing the cost per DALY is equivalent to changing
the cost of vaccine (e.g. an assumption of \$66 per dose of vaccine
and \$100000 per DALY would produce identical results).


\subsection*{GA details}

Point mutations are performed by
incrementing the value of a single gene by $\epsilon$ 
where $\epsilon$ is drawn uniformly
at random from the interval $(-0.1, 0.1)$.


\subsection*{Transmission details}




\begin{figure}

\caption{\footnotesize GA results by generation. 
Each row of panels shows results from a different scenario, with 
70\% infant coverage in (A,B,C),
70\% vaccine efficacy in (D,E,F) , 
45 year mean duration of vaccine derived immunity in (G,H,I),
and 14.5\% leakiness in (J,K,M).
(A, D,G,J) Density maps of disease burden (measured in DALYs) in each generation of strategies. 
More intense color indicates a larger fraction of schedules in that generation of the algorithm.
current generation with the given disease burden. 
(B,E,H,K) Density maps of the number of booster shots administered per person per lifetime each generation of strategies. More intense color indicates a larger fraction of schedules in that generation of the algorithm.
current generation with the given disease burden. 
(B, D, F, H) Surfaces plot showing the average schedule during each algorithm generation (constructed by taking the mean coverage in each age cohort among each 
generation of booster schedules),
with vaccine coverage on the vertical axis plotted against age and algorithm generation.
}

\begin{tabular}{lll}
A&B&C\\
\includegraphics[height=35mm]{"code/figs/2-infection-costs-coverage"}&
\includegraphics[height=35mm]{"code/figs/2-vaccine-doses-coverage"}&
\includegraphics[height=40mm]{"code/figs/2-average strategy-coverage"} \\


D&E&F\\
\includegraphics[height=35mm]{"code/figs/2-infection-costs-efficacy"}&
\includegraphics[height=35mm]{"code/figs/2-vaccine-doses-efficacy"}&
\includegraphics[height=40mm]{"code/figs/2-average strategy-efficacy"} \\


G&H&I\\
\includegraphics[height=35mm]{"code/figs/2-infection-costs-waning"}&
\includegraphics[height=35mm]{"code/figs/2-vaccine-doses-waning"}&
\includegraphics[height=40mm]{"code/figs/2-average strategy-waning"} \\


J&K&L\\
\includegraphics[height=35mm]{"code/figs/2-infection-costs-leakiness"}&
\includegraphics[height=35mm]{"code/figs/2-vaccine-doses-leakiness"}& 
\includegraphics[height=40mm]{"code/figs/2-average strategy-leakiness"} \\



\end{tabular}

\label{fig:BreakdownByGeneration}
\end{figure}




\begin{figure}

\begin{tabular}{ll}
A&B\\
\includegraphics[width=85mm]{"code/figs/4-susceptibles-coverage"}&
\includegraphics[width=85mm]{"code/figs/4-susceptibles-efficacy"}  \\
C&D\\
\includegraphics[width=85mm]{"code/figs/4-susceptibles-waning"} &
\includegraphics[width=85mm]{"code/figs/4-susceptibles-leakiness"}\\



\end{tabular}

\caption{\footnotesize Susceptibilty by age under the most successful booster schedules for
70\% infant coverage (A), 70\% efficacy (B),
a 45 year mean duration of immunity (C),  
leaky immunity preventing 85.5\% of infections (D),
plotted during the first fifty years following the introduction of booster vaccinations.
Color intensity indicates the percentage of individuals in each age group who are susceptible at each time.
In the leaky immunity case, 14.5\% of vaccinated individuals are included in this calculation.
}


\label{fig:Susceptibles}
\end{figure}



\end{document}




















